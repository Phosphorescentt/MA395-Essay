\todo[backgroundcolor=orange]{SEC: Definition of a Network}
Community detection relies on us knowing lots about the underlying structure of a network and to do that we have to understand its properties. This chapter will establish a more formal understanding of networks and will highlight some key properties and methods that we will use to exctract value about community structure later.

\begin{definition}{(Undirected network)}
    Let $V$ be a set of vertices (nodes) and let $E$ be a set of pairs of vertices such that if $e = (x, y) \in E$ then $x, y \in V$. An undirected network is the pair $(V, E) = N$. An edge $e = (x, y) \in E$ is said to join $x$ and $y$ and $y$ to $x$.\label{def:undirected_network}
\end{definition}

The undirected network is the simplest type of network and on its own has interesting enough properties. However, for the sake of example and application, we will also introduce some other types of network that allow for more \emph{directed} models.

\todo[backgroundcolor=orange]{SEC: Different Types of Network}

\begin{definition}{(Directed network)}
    Let $V$ be a set of vertices (nodes) and let $E$ be a set of pairs of vertices such that if $e = (x, y) \in E$ then $x, y \in V$. A directed network is the pair $(V, E) = N$. An edge $e = (x, y) \in E$ is said to join $x$ to $y$. I.e. if $x$ is joined to $y$ then $y$ is not necessarily joined to $x$.
\end{definition}

The intuition for directed graphs, is that edges may only be travelled along in one way. This comes in handy for modelling more intricate systems. The final network type of interest is that of the weighted network.

\begin{definition}{(Weighted network)}
    Let $V$ be a set of vertices (nodes) and let $E$ be a set of triples of the form $V^2 \times \mathbb{R}$ such that if $e = (x, y, w) \in E$ then $x, y \in V$. The value $w$ is said to be the weight of the edge.
\end{definition}
\todo[backgroundcolor=red]{SEC: Interesting Properties of Networks}

Adjacency matrices for different types of graphs \\

Bipartite graphs \\

Paths \\

Components \\

Cut sets \\

The graph laplacian \\
