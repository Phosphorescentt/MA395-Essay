\todo[backgroundcolor=red]{SEC: Introduction to Community Detection}
As allured to in the previous chapters, detecting communities is of great interest and as such there a number of ways to do it. The process of community detection involves analysing the network and finding groups of nodes in the network that are more densely connected amongst themselves than they are to the rest of the network. This notion forms the basis for most community detection methods and algorithms. However, it turns out that it's difficult to come up with a good definition of a community. According to Fortunato, ``In most cases, communitys are algorithmically defined, i.e. they are just the final product of the algorithm without a precise \emph{a priori} definition."\cite[84]{fortunato} In this section, we will discuss the notions of ``community" that underpin a number of interesting methods in community detection before going into detail about the algorithms themselves.

\todo[backgroundcolor=red]{SEC: Background for Community Detection}

% \todo[backgroundcolor=red]{SEC: Traditional Methods of Community Detection}
%
% \todo[backgroundcolor=red]{SEC: Spectral Methods of Community Detection}

\todo[backgroundcolor=red]{SEC: Methods of Community Detection}

\subsection{Louvian Community Detection}

\subsection{Surprise Community Detection}

\subsection{Leiden Community Detection}

\subsection{Walktrap Community Detection}
