\subsection{Personal Learnings}
Throughout the course of writing this essay, I have developed a strong understanding of the foundations of research into community detection. What was most interesting to me was that there is no single strict definition of a community and instead we choose to define communities by the results of algorithms (Section \ref{sec:Community Detection}). This was a shock to me, but also unsurprising. A shock because I had expected that given the utility and power of community detection in understanding real world phenomena we would have a solid way to define what it is we're talking about. Unsurprising because networks are intuitively extremely complex on a large scale and as such identifying macro-properties such as community structure is understandably very hard. The most important personal learning from the writing of this essay was about quality functions and modularity. Of course it is required that there is some quantitative measure of the performance of an algorithm, but until researching them I had underappreciated the intricacy required to make such a measure. Take, for example, the Newman-Girvan modularity from Section \ref{sec:qfs and modularity}. This measure is not as simple as ``calculate the fraction of edges that go between communities". Instead there is a very deliberate design process and set of criteria that a quality function should meet such as normality (the property of being normalised) as well as the ability to omit trivial identifications of a community (such as putting all the nodes in the same community - this is technically a perfect detection of a community, but it doesn't help us understand the structure of the network in any way).

\subsection{Overview}
As a result of the topic being very computational in nature, community detection has scaled with computational power and this means that it is quite a young field. Despite this, a huge amount of research has been done on the topic thus far and Fortunato has done amazing work collating a large amount of the available information\cite{fortunato}. However, as always, there still remain open problems in the field. Recall Section \ref{sec:Community Detection}: In this section, we discussed that currently there is no strict definition of a community and that we typically define communities by the results of community detection algorithms and methods rather than having a definiton a priori. This appears to be the primary open problem in the field right now. Coming up with a strong and well supported definition of a community would let us understand better the quality of our methods and strategies for finding them. Failing this, the next best thing would be to design a comprehensive set of networks and suite of tests that we can give to community detection algorithms to test their efficacy. These networks and tests should obviously cover as many edge cases as possible to ensure that our algorithms work even for the most difficult to detect communities.

It should now be clear that networks and community detection are of great interest to the scientific community for their applications in modelling real world complex systems such as epidemic processes (Section \ref{sec:epidemics}) and social dynamics (Section \ref{sec:zachary_section}). Most importantly of all, we appreciate that community detection may be useful whenever you have a system that can be modelled as a network. For example: understanding the topology of the world wide web\cite{BARABASI200069} or how metabolic networks are organised\cite{Jeong2000}. All of this paired with the fact that our ability to solve large computational problems is ever increasing make communities a topic of interest for the future. I hope that the reader finds themselves with enough information to consider community structure when working on their next problem.
