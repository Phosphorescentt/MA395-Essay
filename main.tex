\documentclass[a4paper]{article}

\usepackage{amsthm}
\usepackage{amsmath}
\usepackage{amsfonts}
\usepackage{caption}
\usepackage{subcaption}
\usepackage{graphicx}
\usepackage[english]{babel}
\usepackage[colorinlistoftodos]{todonotes}
\usepackage[colorlinks=true, allcolors=blue]{hyperref}

\newtheorem{definition}{Definition}
\newtheorem{theorem}{Theorem}

\title{Networks and Community Detection}
\author{Joshua Mankelow}

\begin{document}
\maketitle

\begin{abstract}
    Networks are an example of a rigorous model for analysing and understanding real world complex systems. A very important quality of these network models is the naturally emergent community structure. Community detection allows us to identify clusters in the network that are well connected amongst eachother. If a network is being used to model a real world system then finding this structure has many implications about the behaviour of the system such as predicting the structure of a network after a change in dynamics. In this essay I will discuss some simple methods for community detection in networks and their applications to the analysis and understanding of real world complex systems. \\
\end{abstract}

\newpage

\tableofcontents

\newpage

\section{Introduction to Networks}\label{sec:Introduction to Networks}
Networks are considered as the combination of two separate objects - a set of nodes (vertices) and a set of links (edges) that connect nodes. The idea is to define a structure that can represent a set of things and how they're connected amongst each other. It turns out that this idea is invaluable for modeling real world systems. Examples of such real world systems include \emph{Technological Networks}, \emph{Social Networks}, \emph{Information Networks} and \emph{Biological Networks} as different systems that are modelled by a network.\cite[Contents]{newman10} A brief example of a network would be something like the following: Imagine you and a number of people you speak to regularly are represented as dots (nodes or vertices) on a piece of paper. Then if any two people are friends, the dots representing those people are connected by a line (edge). If you then repeat this process by asking your acquaintances to list all their friends and so on, you will end up with a simple model of a \emph{social network}.

Now that we have this model, it is easy to identify and detect any natural structure that emerges which we can then use to develop an understanding of the behaviour of the real world system that the network represents. The structure that I will explore in this essay is that of \emph{communities}. Generally speaking, communities are subsets of a network that are \emph{densely connected} amongst themselves\label{copy:community_intuition}. I.e. there is some notion of any node within a community being more closely connected to other nodes in the community than nodes outside the community in the average case. Before we dive into the details of communities and detecting them, I wish to provide some motivation by way of example of the kinds of situations that networks can arise and why they are the natural model for the related systems.

\subsection{Social Networks}\label{sec:Social Networks}
To better illustrate the simple notion of a social network mentioned above, I will introduce the canonical community detection example of \emph{Zachary's Karate Club}. Zachary's Karate Club is a dataset where ``The data was collected from the members of a university karate club by Wayne Zachary in 1977.  Each node represents a member of the club, and each edge represents a tie between two members of the club."\cite[Metadata]{konect:2017:ucidata-zachary}. In Figure \ref{fig:zachary_digrams}, there are two different renderings of the Zachary Karate Club. Figure \ref{fig:zachary_spring} shows the network rendered using a ``spring" layout (which is a type of force directed graph drawing\cite{kobourov12})and figure \ref{fig:zachary_circle} shows the network rendered using a ``circle" layout. These different layouts show us different parts of the underlying structure of the network. For example, in Figure \ref{fig:zachary_spring}, it is clear which nodes in the network have the highest degree and which are of lower degree. It also allows you to see some of the community structure in the network. Meanwhile, in Figure \ref{fig:zachary_circle}, it is much easier to see the which nodes edges in the network would need to be removed to disconnect the network in a minimal way. The reason this dataset is the canonical example of community detection is that the question that comes with it is the following: Suppose two members of the club have a disagreement which causes the club to split in two. How does the club split? In Zachary's original paper on the topic, \emph{An Information Flow Model for Conflict in Small Groups}\cite{konect:ucidata-zachary}, Zachary uses community detection techniques to predict how the network will split after the disagreement. Out of 34 people, Zachary correctly predicts how 33 of them will choose a side after the disagreement.

\begin{figure}
    \begin{center}
        \begin{subfigure}[b]{0.45\textwidth}
            \includegraphics[width=\textwidth]{img/zachary_spring}
            \caption{Spring Layout.}
            \label{fig:zachary_spring}
        \end{subfigure}
        \begin{subfigure}[b]{0.45\textwidth}
            \includegraphics[width=\textwidth]{img/zachary_circle}
            \caption{Circle Layout.}
            \label{fig:zachary_circle}
        \end{subfigure}
    \end{center}
    \caption{Two renderings of the Zachary Karate Club network using data from KONECT.cc\cite{konect} and a Python library NetworkX.\cite{SciPyProceedings_11}}
    \label{fig:zachary_digrams}
\end{figure}

There, of course, exist different ways to represent social networks. The way in which you choose to represent them depends on the question you are trying to answer. For example, one might imagine having two types of nodes in a network. One type of node will represent a person and another type of node will represent an event. An edge is drawn between a person and an event if a person attended a given event and person A is considered connected to person B if they both attended the same event. One such example of this is the \emph{Southern Women Dataset}.\cite{konect:southernwomen} This dataset is another example of a community detection problem because after analysis of the data, it was found that women in the group were split into two discrete subgroups.

\begin{figure}
    \begin{center}
        \includegraphics[width=0.95\textwidth]{img/southern_women}
    \end{center}
    \caption{A rendering of the Southern Women Dataset from Newman.\cite[39]{newman10}}
    \label{fig:southernwomen}
\end{figure}

\subsection{Technological Networks}\label{sec:Technological Networks}
As a result of our intensely and digitally connected world, technological networks are of significant interest to researchers. The easiest example to consider is the Internet. The internet consists of many computers all connected by copper or fibre cables which signals are sent through to transmit data. As one might imagine, in the model, the computers are nodes and the cables are the edges. The internet needs to be robust against software and hardware failures and this is where the idea of commnity detection can help us. Saying that we want the internet to be robust is the same as saying that we want every node in the network to be strongly connected to every other node i.e. the number of possible routes between any two nodes is large. This means that, in the philosophy of community detection, we want the internet to act as one large community rather than multiple smaller communities that are loosley connected. An alternative way of looking at this is that once we've managed to identify the communities, we can then figure out which edges and noes are the critical ones that allow passage from one community to another. This allows us to reinforce those edges and nodes to reduce the potential for failure.

Yet another example of a technological network would be the UK Power Grid. Network theory is a useful model here as with the UK Power Grid we're trying to solve exactly the same problem as with the internet --- we want the system to be robust against hardware or software failures.

\subsection{Information Networks}\label{sec:Information Networks}
The most accessible example of an information network is that which is generated by looking back through the citations on a paper recursively. If Paper A references Paper B, then we will draw a directed edge connecting Paper A to Paper B. This will generate a network that shows which papers are referenced by which other papers and how information is reused. Applying community detection to such a network would show us the different academic working groups and perhaps even different fields or subfields of a subject. An exampls of such an analysis is given by Redner.\cite{Redner1998}

Another example of an information network is the World Wide Web which differs from the internet in that it refers to the webpages hosted on the internet rather than the servers and cables themselves. Mapping the world wide web as a network shows us communities of websites that regularly reference each other. The idea of modelling the world wide web as a network has been used by companies like Google to develop tools like PageRank to enable easier browsing of the web\cite{pagerank} and it also allows us to get an understanding of the topology of the web as a whole.\cite{BARABASI200069}

% \begin{figure}
%     \begin{center}
%         % \includegraphics[width=0.95\textwidth]{}
%         \missingfigure{Rendering of Some Web Dataset}
%     \end{center}
%     \caption{A rendering of the Web}
%     \label{fig:Web}
% \end{figure}

\newpage

\section{Properties of Networks}\label{sec:Properties of Networks}
\todo[backgroundcolor=orange]{SEC: Definition of a Network}
Community detection relies on us knowing lots about the underlying structure of a network and to do that we have to understand its properties. This chapter will establish a more formal understanding of networks and will highlight some key properties and methods that we will use to exctract value about community structure later.

\begin{definition}{(Undirected network)}
    Let $V$ be a set of vertices (nodes) and let $E$ be a set of pairs of vertices such that if $e = (x, y) \in E$ then $x, y \in V$. An undirected network is the pair $(V, E) = N$. An edge $e = (x, y) \in E$ is said to join $x$ and $y$ and $y$ to $x$.\label{def:undirected_network}
\end{definition}

The undirected network is the simplest type of network and on its own has interesting enough properties. However, for the sake of example and application, we will also introduce some other types of network that allow for more \emph{directed} models.

\todo[backgroundcolor=orange]{SEC: Different Types of Network}

\begin{definition}{(Directed network)}
    Let $V$ be a set of vertices (nodes) and let $E$ be a set of pairs of vertices such that if $e = (x, y) \in E$ then $x, y \in V$. A directed network is the pair $(V, E) = N$. An edge $e = (x, y) \in E$ is said to join $x$ to $y$. I.e. if $x$ is joined to $y$ then $y$ is not necessarily joined to $x$.
\end{definition}

The intuition for directed graphs, is that edges may only be travelled along in one way. This comes in handy for modelling more intricate systems. The final network type of interest is that of the weighted network.

\begin{definition}{(Weighted network)}
    Let $V$ be a set of vertices (nodes) and let $E$ be a set of triples of the form $V^2 \times \mathbb{R}$ such that if $e = (x, y, w) \in E$ then $x, y \in V$. The value $w$ is said to be the weight of the edge.
\end{definition}
\todo[backgroundcolor=red]{SEC: Interesting Properties of Networks}

Adjacency matrices for different types of graphs \\

Bipartite graphs \\

Paths \\

Components \\

Cut sets \\

The graph laplacian \\

\newpage

\section{Community Detection}\label{sec:Community Detection}
\todo[backgroundcolor=orange]{SEC: Introduction to Community Detection}
As allured to in the previous chapters, detecting communities is of great interest and as such there a number of ways to do it. The process of community detection involves analysing the network and finding groups of nodes in the network that are more densely connected amongst themselves than they are to the rest of the network. This notion forms the basis for most community detection methods and algorithms. However, it turns out that it's difficult to come up with a good definition of a community. In fact, it turns out that ``In most cases, communities are algorithmically defined, i.e. they are just the final product of the algorithm without a precise \emph{a priori} definition."\cite[84]{fortunato} In this section, we will discuss the notions of ``community" that underpin a number of interesting methods in community detection before going into detail about the algorithms themselves.

\todo[backgroundcolor=orange]{SEC: Background for Community Detection}
A simple example of the aforementioned notion comes in the form of inter and intra-cluster densities as discussed by Fortunato.\cite[84]{fortunato}. This idea assumes that we have a network $N$ and a subnetwork $C \subseteq N$ i.e. $C$ is also a network where every node and edge in $C$ is also in $N$, but the converse is not necessarily true. Suppose we want to determine whether $C$ is a community inside $N$. To aid us in our investigation, we can define the following two quantities,

\begin{enumerate}
    \item Intra-cluser density: $\delta_{\text{int}}(C) = \frac{\text{\# of internal edges of C}}{n_c(n_c - 1)/2} $,
    \item Inter-cluser density: $\delta_{\text{ext}}(C) = \frac{\text{\# of external edges of C}}{n_c(n_c - 1)/2} $,
\end{enumerate}

where $n_c = |C|$ and internal and external edges refer to edges originating in $C$ that end inside and outside $C$ respectively. Thus the intra-cluster density is the ratio of edges originating in $C$ that remain in $C$ whilst the inter-cluster density is the ratio of edges originating in $C$ that end outside $C$. Finally, to make sense of these metrics, we introduce a fiducial marker for community structure, the average link density $\delta(N)$. Following the same intuition as before, this quantity represents the ratio of edges that are actually present in the network. Now, if $C$ were a community, we would expect that $\delta_{\text{int}}(C)$ is noticably larger than $\delta(N)$. Similarly, we would expect $\delta_{\text{ext}}(C)$ to be noticably smaller than $\delta(N)$. Getting this result is the goal of most community detection algorithms.

Stricter definitions of communities come in three flavours: local definitions, global definitions and definitions based on vertex similarity. To summarise: local definitions rely on considering the structure of a given subnetwork and perhaps it's immediately adjacent neighbours; global definitions consider the structure of the whole network; and degree similarity relies on some notion of similarity between any two vertices. Each method seeks to formalise our intuition that communities should be strongly connected amongst themselves, but weakly connected to the rest of the network. To extend this intuition, we will refer to Wasserman's summary of the four general properties\cite[251-252]{wasserman_faust_1994} of what he calls ``cohesive subgroups" (but we call communities) that have influenced the formalisations and definitions of the concept in the social network literature. The following are reprinted verbatim:

\begin{enumerate}
    \item The mutuality of ties
    \item The closeness or reachability of subgroup members
    \item The frequency of ties amongst members
    \item The relative frequency of ties amongst subgroup members compared to non-subgroup members
\end{enumerate}

The four points above are written in the parlance of the Social Network Analysis literature. As such, we will reprint them using terms already introduced in this essay.

\begin{enumerate}
    \item Cliques
    \item Closeness or reachability of community members
    \item Frequency of connections between community members
    \item Relative frequency of connections amongst community members compared to non-community members.
\end{enumerate}

Each of these points refers to a different strategy for identifying communities. Paraphrasing Wasserman and using our terminology: Strategies based on cliques require each member of a community to be directly adjacent to each other member of a community; strategies based on closeness or reachability require that each member of a community is reachable from every other member of the same community, but adjacency is not required; strategies based on frequency of connetions between community members require that each member of the community is adjacent to many other members of the community; and strategies based on the relative frequency of connections require that members of the community are more connected amongst each other than they are to the rest of the network.

These four ideas are ordered (roughly) in increasing generality and Fortunato provides an excellent summary of the journey from a rudimentary strategy using cliques to strategies involving fitness measures.\cite[84-85]{fortunato}

Note to later self: I'm kind of interested in references 77 and 78 from Fortunato's writeup. Also just generally I need to give the whole "Local definitions" thing a proper read as I'm sure there's some interesting stuff to go into there.

% In more detail, local definitions can vary a lot but the efficacy can also vary significantly. For example: cliques. Are cliques communities? Yes!

% \todo[backgroundcolor=red]{SEC: Traditional Methods of Community Detection}
%
% \todo[backgroundcolor=red]{SEC: Spectral Methods of Community Detection}

\todo[backgroundcolor=red]{SEC: Methods of Community Detection}

\subsection{Louvian Community Detection}

\subsection{Surprise Community Detection}

\subsection{Leiden Community Detection}

\subsection{Walktrap Community Detection}

\newpage

\section{Applications of Community Detection}\label{sec:Appliations of Community Detection}
\todo[backgroundcolor=red]{Applications of Community Detection}
\todo[backgroundcolor=red]{Figure out an interesting thing to write some of my own code for}

\newpage

\section{Conclusion}
\subsection{Personal Learnings}
Throughout the course of writing this essay, I have developed a strong understanding of the foundations of research into community detection. What was most interesting to me was that there is no single strict definition of a community and instead we choose to define communities by the results of algorithms (Section \ref{sec:Community Detection}). This was a shock to me, but also unsurprising. A shock because I had expected that given the utility and power of community detection in understanding real world phenomena we would have a solid way to define what it is we're talking about. Unsurprising because networks are intuitively extremely complex on a large scale and as such identifying macro-properties such as community structure is understandably very hard. The most important personal learning from the writing of this essay was about quality functions and modularity. Of course it is required that there is some quantitative measure of the performance of an algorithm, but until researching them I had underappreciated the intricacy required to make such a measure. Take, for example, the Newman-Girvan modularity from Section \ref{sec:qfs and modularity}. This measure is not as simple as ``calculate the fraction of edges that go between communities". Instead there is a very deliberate design process and set of criteria that a quality function should meet such as normality (the property of being normalised) as well as the ability to omit trivial identifications of a community (such as putting all the nodes in the same community - this is technically a perfect detection of a community, but it doesn't help us understand the structure of the network in any way).

\subsection{Overview}
As a result of the topic being very computational in nature, community detection has scaled with computational power and this means that it is quite a young field. Despite this, a huge amount of research has been done on the topic thus far and Fortunato has done amazing work collating a large amount of the available information\cite{fortunato}. However, as always, there still remain open problems in the field. Recall Section \ref{sec:Community Detection}: In this section, we discussed that currently there is no strict definition of a community and that we typically define communities by the results of community detection algorithms and methods rather than having a definiton a priori. This appears to be the primary open problem in the field right now. Coming up with a strong and well supported definition of a community would let us understand better the quality of our methods and strategies for finding them. Failing this, the next best thing would be to design a comprehensive set of networks and suite of tests that we can give to community detection algorithms to test their efficacy. These networks and tests should obviously cover as many edge cases as possible to ensure that our algorithms work even for the most difficult to detect communities.

It should now be clear that networks and community detection are of great interest to the scientific community for their applications in modelling real world complex systems such as epidemic processes (Section \ref{sec:epidemics}) and social dynamics (Section \ref{sec:zachary_section}). Most importantly of all, we appreciate that community detection may be useful whenever you have a system that can be modelled as a network. For example: understanding the topology of the world wide web\cite{BARABASI200069} or how metabolic networks are organised\cite{Jeong2000}. All of this paired with the fact that our ability to solve large computational problems is ever increasing make communities a topic of interest for the future. I hope that the reader finds themselves with enough information to consider community structure when working on their next problem.

\newpage

\bibliographystyle{alpha}
\bibliography{bib}

\end{document}
